\documentclass[xcolor={usenames,dvipsnames,svgnames}]{beamer}

\usetheme{mun}
\usepackage{array ,pgfplots}
\usepackage{biblatex}
\bibliography{\jobname}

\title{GAs to Solve the TSP}
\subtitle{Computer Science 3201}
\author{Jacob House // Nabil Miri \\ Omar Mohamed // Hassan El-Khatib}
\date{Fall 2018}

\begin{document}
\begin{frame}[plain]
\titlepage
\end{frame}

\startheads

\section*{Outline}
\begin{frame}
	\tableofcontents
\end{frame}

\section{The Team}
\begin{frame}
\def\arraystretch{2}%  1 is the default, change whatever you need
\begin{tabular}{>{\color{structure}\bfseries}ll}
	Omar Mohamed & \parbox[t]{.75\linewidth}{Project management \\ Programmer} \\
	Nabil Miri & \parbox[t]{.75\linewidth}{Algorithm implementation \\ Debugging} \\
	Jacob House & \parbox[t]{.75\linewidth}{Technical management \\ Code quality control} \\
	Hassan El-Khatib & Programmer
\end{tabular}
\end{frame}

\section{Our Approach}
\subsection{Population Size}
\begin{frame}
\begin{itemize}
	\item<1-> For a route with $n$ cities, we have
	\begin{equation*}
	R := n \cdot (n-1) \cdots 3 \cdot 2 \cdot 1 = n!
	\end{equation*}
	possible routes that cover all cities
	\item<2-> As $n$ grows, so does $R$
	\item<3-> Population size $P$ should also grow with $n$
	\item<4-> We define $P := 2n$ and choose $P$ (not necessarily distinct) permutations of the set $\{0, 1, \ldots, n-1\}$ as the population
\end{itemize}
\end{frame}

\subsection{Mating Pool Size}
\begin{frame}
\begin{itemize}
	\item<1-> Due to the large number of permutations of cities $c_1, c_2, c_3, \ldots c_n$, many of our candidate solutions are likely very low in fitness ({\em i.e., } their total distance is very high)
	\item<2-> Define the mating pool size $M$ to be 
	\begin{equation*}
	M := \left\lfloor\frac{1}{2} \cdot P\right\rfloor
	\end{equation*}
\end{itemize}
\end{frame}

\section{Fitness Scoring}
\subsection{Euclidean Distance in $\mathbb R^2$}
\begin{frame}
\begin{tabular}{ll}
	\hspace*{-1pc}\begin{minipage}{0.5\linewidth}
		\begin{itemize}
			\item<1-> Euclidean distance is computed using the formula
			\begin{equation*}
			\sqrt{(x_2-x_1)^2 + (y_2-y_1)^2}.
			\end{equation*}
			\item<2-> Line $\overrightarrow{AB}$ measures
			\begin{equation*}
			\begin{aligned}
			\left\lVert \overrightarrow{AB} \right\rVert&=\sqrt{(2+1)^2 + (2.5+1.5)^2} \\
			&= \sqrt{9 + 16} \\
			&= 5
			\end{aligned}
			\end{equation*}
		\end{itemize}
	\end{minipage} & 
	\begin{minipage}[t]{0.5\linewidth}
		\begin{tikzpicture}[scale=0.75]
		\begin{axis}[axis y line=middle,axis x line=middle, ymin=-3, ymax=4, xmin=-3, xmax=3]
		\node[label={180:{$A(-1, -1.5)$}},circle,fill,inner sep=2pt] (a) at (axis cs:-1,-1.5) {};
		\node (c) at (axis cs:-1,3) {};
		\node[label={90:{$B(2,2.5)$}},circle,fill,inner sep=2pt] (b) at (axis cs:2,2.5) {};
		\draw[dashed] (a) |- (b);
		\draw (a) -- (b);
		\end{axis}
		\end{tikzpicture}
	\end{minipage}
\end{tabular}
\end{frame}

\subsection{Individual Fitness}
\begin{frame}
\begin{itemize}
	\item<1-> Let $I_m$ with $0 \leqslant m < P$ be a candidate solution of the form 
	\begin{equation*}
	I_m = \left(C_{c_m(\bar 1)}, C_{c_m(\bar 2)}, C_{c_m(\bar 3)}, \ldots, C_{c_m(\bar n)}\right),
	\end{equation*}
	where $c_m\colon \mathbb Z_n \to \{0, 1, 2, \ldots, n-1\}$ is a bijection between congruence classes of indices of the $n$-tuple $I_m$ and the indices of cities. 
	\item<2-> For example, if $n = 7$ and $c_1$ is defined by 
	\begin{gather*}
	c_1\colon \bar 0 \mapsto 4 \qquad c_1\colon \bar 1 \mapsto 6 \qquad c_1\colon \bar 2 \mapsto 2 \qquad c_1\colon 3 \mapsto 1 \\
	c_1\colon \bar 4 \mapsto 3 \qquad c_1\colon \bar 5 \mapsto 5 \qquad c_1\colon \bar 6 \mapsto 0
	\end{gather*}
	then $I_1$ looks like
	\begin{equation*}
	I_1 = \left( C_4, C_6, C_2, C_1, C_3, C_5, C_0 \right).
	\end{equation*}
\end{itemize}
\end{frame}

\begin{frame}
\begin{itemize}
	\item<1-> Then define $I_m$'s overall fitness score $F(I_m)$, to be the summation
	\begin{equation*}
	F(I_m) := \sum_{j = 0}^{n-1} \left\lVert\overrightarrow{C_{c_m(\overline{j})} C_{c_m(\overline{j+1})}}\right\rVert,
	\end{equation*}
	where $\left\lVert\overrightarrow{C_{c_m(\overline{j})} C_{c_m(\overline{j+1})}}\right\rVert$ is the Euclidean distance between city $C_{c_m(\overline{j})}$ and the following city on route $m$, $C_{c_m(\overline{j+1})}$. 
	\item<2-> Hence, the fittest individuals have the {\em lowest} score.
\end{itemize}
\end{frame}

\section{Crossover}
\subsection{The Inver-Over Crossover Operator}
\begin{frame}
For our advanced technique, we have chosen to implement the {\em inver-over} crossover operator which functions according to the following algorithm.
\begin{enumerate}
	\item<1-> Pick an individual $parent_1$ and copy it to $child$
	\item<2-> Then pick two loci from $parent_1$ that depend on another individual $parent_2$ from the population
	\item<3-> Invert everything between these loci in $child$
	\item<4-> Repeat this process with the resulting offspring and another individual $parent_i$ until a stopping condition is reached
\end{enumerate}
\onslide<5->{So {\em inver-over} is a multi-parent crossover operator.}
\end{frame}

\section{Mutation}
\subsection{The Scramble Mutation Operator}
\begin{frame}
The scramble mutation operator, given an individual represented as a sequence of integers, typically performs the following.
\begin{enumerate}
	\item<1-> Picks two loci to form a segment
	\item<2-> Randomly shuffles all information within the selected segments
	\item<3-> Returns the mutated individual
\end{enumerate}
\pause
However, we have slightly modified this operator\ldots
\end{frame}

\begin{frame}
We have added another condition to the operator\only<1>{\ldots}\only<2->{:}\par
\pause 
\begin{quote}
	Define a mutation factor $m \in (0, 1)$. Then the distance between the two chosen loci ({\em i.e.,} the size of the mutation) can be no less than $m$ multiplied by the length of the individual $n$. 
\end{quote}
\pause 
In other words, we have enforced that the product $m \cdot n$ is the infimum of the possible {\em severities} of the mutation.
\end{frame}

\nohead
\section{Demonstration}
\begin{frame}

{\Huge\bfseries Demonstration}
\end{frame}

\startheads
\section{Performance}
\begin{frame}
content...
\end{frame}


\nohead 
\section{Any Questions?}
\begin{frame}

{\Huge\bfseries Any Questions?}
\end{frame}

%\startheads
%\section{References}
%\begin{frame}[allowframebreaks]
%\nocite{*}
%\printbibliography
%\end{frame}
\end{document}
