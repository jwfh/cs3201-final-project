%% BREAK LINES EVERY 80 CHARACTERS TO HELP GIT WITH MERGING
The inver-over operator can be regarded as either a crossover or mutation 
operator because it takes information from other individuals in the GA's
mating pool, yet bases a single offspring off of a single primary parent, much 
the same way a mutation operator is unary, performing mutation on a single
individual \cite{p44}. Throughout this report, the inver-over operator will 
be referenced as a crossover operator as this was the inver-over 
algorithm's place in the GA developed.

\subsection{Usefulness}
As {\em inver-over} embodies aspects of both crossover and mutation 
operators, the operator  is designed to provide middle-ground to 
algorithms relying primarily on crossover for variation (as this is 
computationally expensive) and those relying on mutation for variety,
since this is often ineffective in escaping local minima%
\footnote{Local minima occur when the natural selection in the GA 
	narrows the gene pool towards a ostensibly optimal solution and 
	eliminates individuals that otherwise would evolve to become the 
	true optimal solution.} \cite{p44}.


\subsection{Algorithm}
Given a pool of individuals of length $n$ represented each as a sequence
\begin{equation*}
I_m = \langle C_{c_m(1)}, C_{c_m(2)}, \ldots, C_{c_m(n)} \rangle,
\end{equation*}
